\subsection{Vanka's approach}
\label{subsec:vanka_approach}

The \acrshort{scgs} method is based on the idea of solving the equations in a coupled way.

This means that for each iteration, we will obtain the correction for each of the 5 variables simultaneously.
To make this possible, we need to introduce a new matrix, called the \textbf{Vanka matrix \cite{VANKA1986138}}.

Considering the system of equations \ref{eq:residual_system}, we can write them in a matrix form as:

\begin{equation}
    \begin{bmatrix}
        (A_P^u)_{i-1,j} & 0             & 0               & 0             & \Delta y  \\
        0               & (A_P^u)_{i,j} & 0               & 0             & -\Delta y \\
        0               & 0             & (A_P^v)_{i,j-1} & 0             & \Delta x  \\
        0               & 0             & 0               & (A_P^v)_{i,j} & -\Delta x \\
        -\Delta y       & \Delta y      & -\Delta x       & \Delta x      & 0
    \end{bmatrix}
    \begin{bmatrix}
        u'_{i-1,j} \\
        u'_{i,j}   \\
        v'_{i,j-1} \\
        v'_{i,j}   \\
        p'_{i,j}
    \end{bmatrix}
    =
    \begin{bmatrix}
        R_{i-1,j}^u \\
        R_{i,j}^u   \\
        R_{i,j-1}^v \\
        R_{i,j}^v   \\
        R_{i,j}^c
    \end{bmatrix}
    \label{eq:vanka_matrix}
\end{equation}

As we can see, the Vanka matrix is a $5 \times 5$ matrix, and it's a function of the coefficients of the discretized equations, and of the grid spacing.

Our unknown is the vector of the corrections, being the $A_P^\phi$ coefficients and the $R^\phi$ residuals known from the previous iteration.

Given that, we can solve the system of equations \ref{eq:vanka_matrix} to obtain the corrections for the velocity and pressure components at each cell of the domain.

To do so, we must invert the Vanka matrix, and multiply it by the residuals vector:

\begin{equation}
    \begin{bmatrix}
        \phi'
    \end{bmatrix}
    =
    \begin{bmatrix}
        A^\phi
    \end{bmatrix}^{-1}
    \begin{bmatrix}
        R^\phi
    \end{bmatrix}
\end{equation}

As stated in \cite{VANKA1986138}, the inverse of the Vanka matrix can be computed analytically using the following algorithm:

\begin{gather}
    r_i = \frac{a_{5i}}{a_{ii}}, \quad i = 1, 2, 3, 4 \\
    DEN = \sum_{i=1}^{4} a_{5i} r_i \\
    x_5 = \frac{\left[\sum_{i=1}^{4} r_i b_i - b_5\right]}{DEN} \\
    x_i = \frac{b_i - a_{i5} x_5}{a_{ii}}, \quad i = 1, 2, 3, 4
\end{gather}

Where $a_{ij}$ are the elements of the Vanka matrix $A^\phi$, $b_i$ are the elements of the residuals vector $R^\phi$, and $x_i$ are the elements of the corrections vector $\phi'$.

In case we want to apply an under-relaxation factor to the corrections, we can modify Vanka's Matrix as follows:

\begin{equation}
    \begin{bmatrix}
        \frac{(A_P^u)_{i-1,j}}{\alpha_u} & 0                              & 0                                & 0                              & \Delta y  \\
        0                                & \frac{(A_P^u)_{i,j}}{\alpha_u} & 0                                & 0                              & -\Delta y \\
        0                                & 0                              & \frac{(A_P^v)_{i,j-1}}{\alpha_v} & 0                              & \Delta x  \\
        0                                & 0                              & 0                                & \frac{(A_P^v)_{i,j}}{\alpha_v} & -\Delta x \\
        -\Delta y                        & \Delta y                       & -\Delta x                        & \Delta x                       & 0
    \end{bmatrix}
\end{equation}

Where $\alpha_u$ and $\alpha_v$ are the under-relaxation factors for the velocity components $u$ and $v$ respectively.