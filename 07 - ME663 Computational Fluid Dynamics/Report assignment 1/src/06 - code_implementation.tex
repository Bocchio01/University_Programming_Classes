\section{Code implementation}
\label{sec:code_implementation}

When it comes to the implementation of the \acrshort{scgs} algorithm, lots of details need to be taken into account, such as the data structures, the memory allocation, the data exchange between the different modules of the code, and so on.

% In the following, we report the core of the \acrshort{scgs} algorithm, both in a flowchart and in a \texttt{C} implementation.
% In the following, we report the core of the \acrshort{scgs} algorithm in a \texttt{C} implementation.

The main blocks of the \acrshort{scgs} algorithm can be found in the appendix section \ref{sec:appendix} of this document.

The complete codebase is available on GitHub at \url{https://github.com/Bocchio01/CFD_Simulation_Engine}, along with its technical documentation.

% \begin{figure}[H]
%     \centering

%     \tikzstyle{decision} = [diamond, draw, fill=red!20, text width=4.5em, text badly centered, node distance=3cm, inner sep=0pt]
%     \tikzstyle{block} = [rectangle, draw, fill=blue!20, text width=5em, text centered, rounded corners, minimum height=4em]
%     \tikzstyle{line} = [draw, -latex']

%     \begin{tikzpicture}[node distance = 2cm, auto]
%         % Place nodes
%         \node [block] (init) {Allocate Memory};
%         \node [block, below of=init] (reset) {Reset SCGS};
%         \node [block, below of=reset] (compose) {Compose System Matrix};
%         \node [block, below of=compose] (noslip) {Apply Normal No-Slip BCs};
%         \node [block, below of=noslip] (solve) {Solve System Matrix};
%         \node [block, below of=solve] (correct) {Apply Corrections};
%         \node [block, below of=correct] (residuals) {Update Residuals};
%         \node [block, below of=residuals] (noslipt) {Apply Tangential No-Slip BCs};
%         \node [decision, below of=noslipt] (converged) {Converged?};
%         \node [block, below of=converged, node distance=3cm] (output) {Output Solution};
%         \node [block, left of=converged, node distance=4cm] (update) {Update Iteration};

%         % Draw edges
%         \path [line] (init) -- (reset);
%         \path [line] (reset) -- (compose);
%         \path [line] (compose) -- (noslip);
%         \path [line] (noslip) -- (solve);
%         \path [line] (solve) -- (correct);
%         \path [line] (correct) -- (residuals);
%         \path [line] (residuals) -- (noslipt);
%         \path [line] (noslipt) -- (converged);
%         \path [line] (converged) -- node [near start] {no} (update);
%         \path [line] (update) |- (reset);
%         \path [line] (converged) -- node {yes}(output);

%         % Draw three diamonds for the first loop
%         \node [decision, left of=noslip, node distance=6cm] (loop1a) {Loop 1};
%         \node [decision, below of=loop1a, node distance=3cm] (loop1b) {Loop 2};
%         \node [decision, below of=loop1b, node distance=3cm] (loop1c) {Loop 3};
%         \path [line] (noslip) -| node [near start] {} (loop1a);
%         \path [line] (loop1a) -- node [near start] {} (loop1b);
%         \path [line] (loop1b) -- node [near start] {} (loop1c);
%         \path [line] (loop1c) |- node [near start] {} (solve);
%     \end{tikzpicture}

%     \caption{Flowchart of the \acrshort{scgs} algorithm.}
%     \label{fig:scgs_algorithm}

% \end{figure}

