\section{Introduction}
\label{sec:introduction}

Compressible fluids are characterized by significantly changes in density at pressure variations.
This is typically the case for gases, which can be compressed or expanded, or for liquids at high speed or high pressure.

The metrics used to describe the compressibility of a fluid are the speed of sound $a$ and the Mach number $M$.
The speed of sound is the speed at which waves propagate through the fluid, and it is given by the equation:

\begin{equation}
    a = \sqrt{\frac{\gamma p}{\rho}}
\end{equation}

Where $\gamma$ is the adiabatic index, $p$ is the pressure, and $\rho$ is the density.

The Mach number is the ratio of the speed of an object to the speed of sound in the fluid, and it is given by the equation:

\begin{equation}
    M = \frac{u}{a}
\end{equation}

Where $u$ is the velocity of the object.

In this assignment, we are going to study the compressible fluid using the Euler equations, which are a set of equations that describe the conservation of mass, momentum, and energy in a fluid.
The Euler equations are given by the following system of partial differential equations:

\begin{equation}
    \frac{\partial U}{\partial t} + \nabla \cdot F = 0
\end{equation}

Where $U$ is the vector of conserved variables, and $F$ is the vector of fluxes.

If we consider a 1D flow, the Euler equations can be written as:

\begin{equation}
    \frac{\partial U}{\partial t} + \frac{\partial F}{\partial x} = 0
    \label{eq:euler_equations}
\end{equation}

After some calculation that involve the explicit form of $F(U)$, we can observe that Euler equations are a hyperbolic system of partial differential equations, which means that they have a wave-like solution.

The final request for this assignment is to solve numerically Equation \ref{eq:euler_equations} using the flux-vector splitting method, which consists of splitting the flux vector based on the eigenvalues of the Jacobian matrix of the flux vector.

In the following, we will derive the Jacobian of the flux vector (Section \ref{sec:Q1}), the eigenvalues and eigenvectors of the matrix (Section \ref{sec:Q2}), and the splitted flux vectors formulation (Section \ref{sec:Q3}), highlighting its dependence on the eigenvalues of the Jacobian matrix.

In section \ref{sec:Q4}, we will focus on how to compute the eigenvalues of the system, reporting both the Steger \& Warming formulation \cite{Steger-Warming} and the van Leer formulation \cite{van-Leer}.

Finally, we will implement a 1D solver for the Euler equations using the flux-vector splitting method and compare the results with the exact solution (Section \ref{sec:Q5}).
