\subsection{Fluent setup}
\label{subsec:fluent_setup}

The next step in the CFD simulation is to set up \texttt{Fluent} solver.

Here, many configurations could be set, such as the turbulence model, the solver type, the discretization scheme, etc.
Given our inexperience with CFD simulations, we opted for the default settings for most of the parameters with some exception that are reported in the following subsections.


\subsubsection{Turbulence model}
\label{subsubsec:turbulence_model}

One of the most important parameters to set in the CFD simulations is the turbulence model.

The turbulence model is a mathematical model used to predict the effects of turbulence on the flow field.
In our case, we opted for the \textit{k-$\omega$ SST} model, which is a widely used model for external aerodynamics simulations.

The SST \textit{k-$\omega$} model is a two-equation eddy-viscosity model that basically combines both the \textit{k-$\omega$} and \textit{k-$\epsilon$} models.
The use of this model allows for a good prediction of the flow field around the rocket, especially in the near-wall regions and transitional flows.

To do so, we use a \textit{k-$\omega$} model in the inner parts of the boundary layer, making the model directly usable all the way down to the wall through the viscous sub-layer, hence the SST \textit{k-$\omega$} model can be used as a Low-Re turbulence model without any extra damping functions.
The SST formulation also switches to a \textit{k-$\epsilon$} behaviour in the free-stream, avoiding the common \textit{k-$\omega$} problem that the model is too sensitive to the inlet free-stream turbulence properties.

Refer to Menter \cite{Menter1994TwoequationET} for more details on the SST \textit{k-$\omega$} model.


\subsubsection{Report definition}
\label{subsubsec:report_definition}

In the \texttt{Fluent} solver, it is possible to define the reports that will be generated during the simulation.

In our case, we have defined a custom report for the computation of the drag coefficient.
The only parameter that we had to modify to enhance the accuracy, was the reference area.

For a simple model rocket, the reference area can be considered as the largest cross-sectional area of the body tube.
In our case:

\begin{equation}
    A_{\text{ref}} = \pi \cdot \left( \frac{D}{2} \right)^2 = 0.0009079203 \quad [m^2]
\end{equation}


