\subsection{Compressibility effects}
\label{subsec:compressibility_effects}

Before moving on, we need to decide whether to take into account the compressibility effects in the simulations or not.

It's well known from literature and experimental data that compressibility effects start to become important when the flow speed reaches a significant fraction of the speed of sound.
In particular, the Mach number of the flow is a good indicator of the importance of the compressibility effects.
The condition that distinguishes between incompressible and compressible flows is given by:

\begin{equation}
    Ma = \frac{v}{c} = \begin{cases}
        < 0.3 & \text{Incompressible flow} \\
        > 0.3 & \text{Compressible flow}
    \end{cases}
    \label{eq:mach_number}
\end{equation}

Where $Ma$ is the Mach number, $v$ is the flow speed and $c$ is the speed of sound in the fluid.

To determine whether to take into account the compressibility effects or not, we need to make some hand calculations based on the whether condition present when the rocket was flying.
In Table \ref{tab:conditions} we report the relevant parameters.

\begin{table}[H]
    \centering
    \begin{tabular}{|c|c|}
        \hline
        \textbf{Parameter} & \textbf{Value} \\
        \hline
        Altitude           & $1444m$        \\
        Temperature        & $15^{\circ}C$  \\
        \hline
    \end{tabular}
    \caption{Conditions present during the rocket flight.}
    \label{tab:conditions}
\end{table}

Considering air as an ideal gas, the definition of the speed of sound is given by the following equation:

\begin{equation}
    c = \sqrt{\gamma \frac{p}{\rho}} = \sqrt{\gamma \bar{R} T}
    \label{eq:speed_of_sound}
\end{equation}

Where $\gamma$ is the specific heat ratio, $\bar{R}$ is the specific gas constant and $T$ is the temperature of the fluid.

The specific heat ratio for air is $\gamma = 1.4$ and the specific gas constant considering dry air can be computed as:

\begin{equation}
    \bar{R} = \frac{R}{M_{mol}} = \frac{8.314}{\frac{28.96}{1000}} = 287.05 \frac{J}{kg \cdot K}
\end{equation}

Where $R$ is the universal gas constant and $M_{mol}$ is the molar mass of dry air.

Plugging the values into Equation \ref{eq:speed_of_sound}, we get:

\begin{equation}
    c = \sqrt{1.4 \cdot 287.05 \cdot (273.15 + 15)} = 340.3 \frac{m}{s}
\end{equation}

From the rocket flight data, we know that the rocket reached a maximum speed of around $130m/s$.

Plugging the values into Equation \ref{eq:mach_number}, we get:

\begin{equation}
    Ma = \frac{130}{340.3} = 0.382
\end{equation}

The maximum Mach number is $0.382$, which is above the threshold of $0.3$.
For this reason, we decided to take into account the compressibility effects in the simulations.
To do so, we modified the \textit{Material} properties in \texttt{Fluent}, setting the model of the air to \textit{Ideal Gas} instead of \textit{Incompressible}.

However, we believe that this choice is not critical for the results of the simulations, given that the maximum Mach number was reached only for a short period of time during the flight, while for the rest of it $Ma$ was below the threshold.
