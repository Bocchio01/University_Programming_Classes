\section{Methodology}
\label{sec:methodology}

This project consists of three main steps: CAD modeling, CFD simulation, and analysis of the results.
In this section, we will briefly describe each step and explain the tools and techniques used to carry out the analysis.

\paragraph{CAD modeling}

The first step in the analysis is to create a CAD model of the rocket.
The model is used to define the geometry that will be analyzed in the CFD simulations.

For this purpose, we opted for \texttt{CATIA V5}, a computer-aided design (CAD) software package from \texttt{Dassault Systèmes}.


\paragraph{CFD simulation}

The second step in the analysis is to set up and perform the CFD simulations of the model rocket.
In this project, CFD simulations are used to calculate the drag coefficient ($C_d$) and obtain insights into the aerodynamic performance of the rocket.
It has to be noted that simulations will be performed in a steady-state regime, and considering as velocity of the fluid $85\%$ of the peak one that the rocket reaches during the real flight.
Considering the drag coefficient as a constant value across the whole range of velocities it's a strong assumption, and should be avoided in a real-world scenario.

The CFD simulations were performed using \texttt{Ansys Fluent}, a computational fluid dynamics (CFD) software package from \texttt{Ansys Inc.}.
This was our first time using a commercial CFD software package, and we found it to be a powerful tool for simulating complex fluid flow problems.


\paragraph{Analysis and validation of the results}

The final step is to try to validate the results of the CFD simulations.

To do so, we will use \texttt{MATLAB} to code a simple physics model of the rocket's flight and compare it against two different sources of data:

\begin{itemize}
    \item \texttt{OpenRocket} simulations
    \item Actual flight data of the rocket, collected during a previous launch
\end{itemize}

\texttt{OpenRocket} is an open-source software for model rocket design and simulation.
It allows to select the engine and set the material properties of the rocket.
By considering a mean value of the drag coefficient, the software is then able to predict the flight and the apogee altitude.

The actual flight data of the rocket were collected using a barometric altimeter.
Unfortunately, the data are very a few, due to the short duration of the flight and the low sampling rate of the altimeter.
However, as we will see, they still can be useful to validate the CFD simulations.

The \texttt{MATLAB} code used to perform the analysis is both available in the appendix (Appendix \ref{sec:appendix}) and can also be found among the file attached to this report.