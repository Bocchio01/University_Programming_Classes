\section{Conclusions}
\label{sec:conclusions}

The present work aimed at investigating the drag coefficient of a model rocket using computational fluid dynamics.

The rocket was at first modeled in an external CAD software and then imported into Ansys Fluent for the CFD simulation.
The simulation was carried out using the $k-\omega$ SST turbulence model and in a steady-state regime and the drag coefficient ($C_d$) was calculated at $85\%$ of the speed maximum speed reached by the rocket during the real world flight.

The results obtained from the CFD simulation were compared to the flight data and the simulation from \texttt{OpenRocket}, showing a good agreement with both of them after some adjustments to the thrust curve.

Moreover, the drag coefficient was found to be $C_d = 0.86$, which is in line with the expected values for a model rocket.
This value is also in line with the one obtained from the flight data and the one from \texttt{OpenRocket}.

The present work showed that CFD simulations can be a powerful tool to investigate the aerodynamic behavior of model rockets and that the results obtained can be potentially used to optimize the design iterating backward and forward between the CAD model and the CFD simulation.