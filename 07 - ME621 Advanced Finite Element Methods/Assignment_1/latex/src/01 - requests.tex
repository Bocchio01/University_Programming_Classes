\section{Requests}
A system of two aluminum bars of the same material is shown in the following figure.
The system is subjected to two external loads, $P_x$ and $P_y$, at joint B.
A and C are connected to pinned supports.

\begin{figure}[h]
    \centering
    \begin{tikzpicture}[scale=3]

        \coordinate (A) at (0,0.5);
        \coordinate (B) at (3,0.5);
        \coordinate (Bf) at (3.5,1);
        \coordinate (C) at (3,0);

        % Joint names
        \node at (A) [above, left] {A};
        \node at (B) [below, above] {B};
        \node at (C) [below, left] {C};

        % Initial position
        \draw (A) -- (B) node[midway, below] {$L_1$};
        \draw (C) -- (B) node[midway, left] {$L_2$};

        % Deformed position
        \draw[dashed] (A) -- (Bf)node[midway, above] {$l_1$};
        \draw[dashed] (C) -- (Bf)node[midway, right] {$l_2$};

        % Labels
        \pic [draw, ->, "$\alpha$", angle radius=2cm] {angle = B--A--Bf};
        \pic [draw, <-, "$\beta$", angle radius=0.7cm] {angle = Bf--C--B};

        % Support at the end of Beam 1
        \draw[fill] (0,0.5) circle (0.03);

        % Support at the end of Beam 2
        \draw[fill] (3,0) circle (0.03);

        % Arrows at coordinate Bf
        \draw[->] (Bf) -- ++(0.3, 0) node[right] {$\vec{P_x}$};
        \draw[->] (Bf) -- ++(0, 0.3) node[above] {$\vec{P_y}$};
        \draw[->] (B) -- (Bf) node[midway, above] {$\vec{u}$};

    \end{tikzpicture}
    \caption{Problem representation}
    \label{fig:problem_representationy}
\end{figure}

The problem asks to:

\begin{itemize}
    \item Obtain the external loads $P_x$ and $P_y$ as a function of horizontal and vertical displacements at point B (namely $u$ and $v$).
    \item Determine the displacements in both $x$ and $y$ directions for $1000$ load increments of $+5\text{N}$ for both $P_x$ and $P_y$ (from zero).
    \item Find the displacement of point B after the final increment.
\end{itemize}

\noindent Write a MATLAB code with a convergence error of $10^-5$ to numerically solve the problem.
Use a combination of (a) Euler and N-R, and (b) Euler and modified N-R.
Also plot the resultant force versus the resultant displacement.

\noindent Use the Green strain measure:

\begin{equation}
    \epsilon = \frac{l^2 - L^2}{2L^2}
\end{equation}

\begin{table}[H]
    \centering
    \begin{tabular}{|c|c|c|}
        \hline
        \textbf{Parameter} & \textbf{Value} & \textbf{Unit} \\ \hline
        $E_1 = E_2 = E$    & $70$           & $\text{GPa}$  \\ \hline
        $L_1$              & $3$            & $\text{m}$    \\ \hline
        $L_2$              & $0.5$          & $\text{m}$    \\ \hline
        $A_1 = A_2 = A$    & $0.0001$       & $\text{m}^2$  \\ \hline
    \end{tabular}
    \caption{Parameters of the system}
    \label{tab:parameters_of_the_system}
\end{table}
