\section{Motivations}

\begin{frame}{Disturbances effect of environmental condition on SHM}

    In case of axial-load beams (tie-rods), studies has highlighted that \textbf{temperature variations can cause greater changes to structural vibration than the presence of damage itself}.

    The transverse vibration in a tensioned beam is described by the following equation\footnotemark[1]:
    \begin{equation}
        w(\xi, t) = \left[ A \sin(\gamma_1 \xi) + B \cos(\gamma_1 \xi) + C \sin(\gamma_2 \xi) + D \cos(\gamma_2 \xi)\right] E \cos(\omega t + \phi)
    \end{equation}

    Where:
    \begin{equation}
        \gamma_1 = \sqrt{\frac{N - \sqrt{N^2 + 4EJ \rho A \omega^2}}{2EJ}} \qquad \gamma_2 = \sqrt{\frac{N + \sqrt{N^2 + 4EJ \rho A \omega^2}}{2EJ}}
    \end{equation}

    Notice that $N = N(Temperature) = N_0 + k (T - T_0)$, with $k \approx -60 \frac{N}{^\circ C}$.

    \footnotetext[1]{A full derivation of the equation can be found in the appendix.}

\end{frame}