\section{Conclusions}

\begin{frame}{Method comparison}

    Overall, both the proposed approaches has shown to be effective in detecting damage in the tie-rods.

    However, the use of the PCA methods offers some non-negligible advantages:

    \begin{itemize}
        \item It's more robust in isolate the damage features from other sources of variability.
        \item It doesn't require a training set that includes all the possible environmental conditions, thus eliminating the need for a long data sampling campaign.
    \end{itemize}

\end{frame}



\begin{frame}{Future work and preliminary solutions}

    Regarding the PCA method, some possible future developments and solutions spotted during the research are:

    \begin{table}
        \centering
        \begin{tabular}{l|l}
            \textbf{Future work}  & \textbf{Preliminary solution}                                                        \\
            \hline
            Automatic PCs removal & Set a threshold based on an RMS of each PCs and remove the PC that exceed this limit \\
            \hline
        \end{tabular}
    \end{table}

\end{frame}