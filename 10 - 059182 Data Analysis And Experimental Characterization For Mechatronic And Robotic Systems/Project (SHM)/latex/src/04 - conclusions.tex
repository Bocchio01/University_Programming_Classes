\section{Conclusions}

\begin{frame}{Conclusions}

    Overall, both the proposed approaches have shown to be effective in performing SHM over simple structure like tie-rods, with the capability of detecting damage even in the presence of environmental variability.

    \vspace{9pt}

    However, the use of the \textbf{PCA methods} offers some non-negligible advantages:

    \begin{itemize}
        \item It's \textbf{more robust in isolate the damage features} from other sources of variability.
        \item It doesn't require a training set that includes all the possible environmental conditions, thus eliminating the need for a long data sampling campaign.
    \end{itemize}

    \vspace{9pt}

    Moreover, parameters like the baseline set length $b$ and the observation window length $n$ must be carefully chosen to obtain an optimal performance of the method.

\end{frame}



% \begin{frame}{Future work and preliminary solutions}

%     Regarding the PCA method, some possible future developments and solutions spotted during the research are:

%     \begin{table}
%         \centering
%         \begin{tabular}{l|l}
%             \textbf{Future work}  & \textbf{Preliminary solution}                                                        \\
%             \hline
%             Automatic PCs removal & Set a threshold based on an RMS of each PCs and remove the PC that exceed this limit \\
%             \hline
%         \end{tabular}
%     \end{table}

% \end{frame}