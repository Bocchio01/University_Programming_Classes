\subsection{Acceleration approach}
\label{subsec:acceleration_approach}

The acceleration approach instead, consists of assembling the global force vector $\mathbf{F}$ based on the idea that the structure (since it has been discretized), can be seen as a system of masses connected by springs and dampers.
Because of this, a distributed acceleration load can be directly transformed into a series of nodal forces acting by solving the second Newton's law for each node of the structure.

In particular, the nodal forces can be computed as:

\begin{equation}
    \mathbf{F} = M \mathbf{acceleration}
\end{equation}

Where $M$ is the mass matrix of the structure and $\mathbf{acceleration}$ is the vector of accelerations acting on each degree of freedom of the structure.

Finally, we can solve the equation of motion \ref{eq:static_responses} to obtain the displacements of the structure.

In \texttt{MATLAB}, the acceleration approach can be implemented as follows:

\begin{lstlisting}[language=Matlab]

    % Acceleration approach.
    x_dot_dot = zeros(ndof, 1);
    x_dot_dot(idb(:, 2)) = -9.81;

    F_nodal = M * x_dot_dot;

    X_gravity_acceleration_approach = K_FF \ F_nodal(1:ndof);

\end{lstlisting}
