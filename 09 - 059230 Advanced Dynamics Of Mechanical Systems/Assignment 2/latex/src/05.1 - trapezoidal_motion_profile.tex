\subsection{Trapezoidal motion profile}
\label{subsec:trapezoidal_motion_profile}

As for the motion type of the mass along the $x$ axis, we can consider a trapezoidal velocity profile, with a constant acceleration phase, a constant velocity phase, and a constant deceleration phase.
In particular, if we consider the total distance covered by the mass as $L = 24[m]$ and a total time of $T = 20[s]$, we can compute both the accelerations ($a_{acc} = |a_{dec}| = a$), velocity ($v_{max}$), and the time duration of each phase ($t_1$, $t_2$ and $t_3$):

\begin{equation}
    \begin{cases}
        L = \frac{1}{2} \cdot a \cdot t_1^2 + v_{max} \cdot t_2 + \frac{1}{2} \cdot |a| \cdot t_3^2 \\
        T = t_1 + t_2 + t_3
    \end{cases}
\end{equation}

By solving the system of equations, we can find the following values:

\begin{equation}
    \begin{cases}
        a = 0.25 [m/s^2]  \\
        v_{max} = 2 [m/s] \\
        t_1 = 8 [s]       \\
        t_2 = 4 [s]       \\
        t_3 = 8 [s]
    \end{cases}
\end{equation}

Once the position of the load is computed, we must compute the equivalent nodal forces that is applied to the structure because of the load.
By neglecting any momentum nodal load, and by considering just the vertical nodal force, we can compute the equivalent nodal force by solving the following equations:

\begin{align}
    \xi           & = \frac{x - x_1}{x_2 - x_1} \\
    F_{eq, left}  & = m \cdot g \cdot \xi       \\
    F_{eq, right} & = m \cdot g \cdot (1 - \xi)
    \label{eq:equivalent_nodal_force}
\end{align}

Where $x$ is the position of the mass, $x_1$ and $x_2$ are the position of the two nodes that are closest to the mass, and $g$ is the gravity acceleration.

Once the nodal forces are known for each time step, we can compute the nodal displacements by solving the partial differential equation that describes the motion of the structure:

\begin{equation}
    M \cdot \ddot{u(t)} + C \cdot \dot{u(t)} + K \cdot u(t) = F_{eq}(t)
    \label{eq:dynamic_equation}
\end{equation}

Where $M$, $C$ and $K$ are the mass, damping and stiffness matrices of the structure, $u$ is the nodal displacement vector, and $F_{eq}$ is the equivalent nodal force vector.

To solve the equation, we must rely on a numerical method, such as the `Runge-Kutta' method, that allows us to solve the equation for each time step.
