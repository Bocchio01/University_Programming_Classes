\subsection{Initial conditions}
\label{subsec:initial_conditions}

One important aspect about this type of analysis is the initial conditions of the structure.

In particular, we can consider two different initial conditions:

\begin{itemize}
    \item \textbf{Initial condition 1}: the structure is in its undeformed configuration, and the load is suddenly added to the structure.
    \item \textbf{Initial condition 2}: the structure is in its deformed configuration due to the static loads of the mass, which then starts to move along the $x$ axis.
\end{itemize}

As we can imagine, the first set of initial conditions will lead to a much more complex dynamic response since it will be the superposition of the fast dynamic response due to the sudden addition of the load (oscillatory behavior) and the slow dynamic response due to the moving load.
On the other hand, the second set of initial conditions will lead to an almost steady dynamic response, since the structure starts from a condition much more near (in terms of displacements) to the final condition.