\section{Implementation}
\label{sec:implementation}

Given that most of the FEM problem is based on the evaluation of integrals, it's now useful to report in the following the implementation the Gaussian quadrature integration method.

\begin{lstlisting}[language=Matlab]
function result = method_gaussian_quadrature_integration(integrand_func, fem)
    % METHOD_GAUSSIAN_QUADRATURE_INTEGRATION
    %   This function computes the integral of a given function using the
    %   Gaussian quadrature method.
    %
    %   result = METHOD_GAUSSIAN_QUADRATURE_INTEGRATION(integrand_func, fem)
    %
    %   integrand_func: function handle to the integrand
    %   fem: struct containing the number of Gauss points required
    %
    %   result: value of the integral


    switch fem.N_Gauss_points
    case 1
        weights = [2];
        roots = [0];
    case 2
        weights = [1, 1];
        roots = [-1/sqrt(3), 1/sqrt(3)];
    case 3
        weights = [5/9, 8/9, 5/9];
        roots = [-sqrt(3/5), 0, sqrt(3/5)];
    otherwise
        error('Order of Gaussian quadrature not implemented (too high)')
    end

    result = 0;
    for root_idx = 1:length(roots)

        result = result + integrand_func(roots(root_idx)) * weights(root_idx);

    end
end
\end{lstlisting}

The core algorithm of the explicit Lagrangian FEM is implemented in the 'Main.m' file attached to this report.