\subsection{Constitutive model}
\label{subsec:constitutive_model}

Our final major step in the solution of the problem at hand is the definition of the constitutive model.

For the given problem, we are asked to use Truesdell objective stress rate to define the constitutive model.

In continuum mechanics, an objective stress tensor (as for example the Truesdell one) is used to describe the rate of change of stress within the material undergoing deformation in a way that it doesn't depend on the observer's frame of reference.

The Truesdell objective stress tensor is defined as:

\begin{equation}
    {\sigma}^{oT} = C : D = \dot{\sigma} - L \sigma - \sigma L^T + tr(L) \sigma
    \label{eq:truesdell_stress_rate}
\end{equation}

where $\sigma$ is the Cauchy stress tensor, $C$ is the stiffness or elasticity tensor, $D$ is the (symmetric) rate of deformation, and $L$ is the (asymmetric) velocity gradient tensor.

In particular, the following definition can be used inside Equation \ref{eq:truesdell_stress_rate}:

\begin{equation}
    C = \begin{bmatrix}
        2\mu + \lambda & \lambda        & 0   \\
        \lambda        & 2\mu + \lambda & 0   \\
        0              & 0              & \mu
    \end{bmatrix} = \frac{E}{(1 - \nu^2)} \begin{bmatrix}
        1   & \nu & 0                 \\
        \nu & 1   & 0                 \\
        0   & 0   & \frac{1 - \nu}{2}
    \end{bmatrix} \quad \text{Considering plane stress}
    \label{eq:stiffness_tensor}
\end{equation}

\begin{equation}
    L = \vec{\nabla} \otimes \vec{v} = \frac{\partial \vec{v}}{\partial \vec{x}} = \frac{\partial (\vec{v}_{node} N)}{\partial \vec{x}} = \vec{v}_{node} B^T
    \label{eq:velocity_gradient_tensor}
\end{equation}

\begin{equation}
    D = \frac{1}{2} (L + L^T) = B \vec{v}_{node}
    \label{eq:rate_of_deformation}
\end{equation}

where $\mu$ is the shear modulus, $\lambda$ is the Lamé constant, $E$ is the Young's modulus, and $\nu$ is the Poisson's ratio.

Given all the definitions above, we can observe that the only unknown in Equation \ref{eq:truesdell_stress_rate} is the rate of Cauchy stress tensor $\dot{\sigma}$, that can be computed as:

\begin{equation}
    \dot{\sigma} = C : D + L \sigma + \sigma L^T - tr(L) \sigma
    \label{eq:rate_of_stress}
\end{equation}

The result from Equation \ref{eq:rate_of_stress} can be used to compute the stress tensor at the next time step in the simulation given that:

\begin{equation}
    \sigma_{n+1} = \sigma_n + \dot{\sigma}_{n+1/2} \Delta t_{n+1/2}
\end{equation}