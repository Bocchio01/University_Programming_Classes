\section{Methodology}
\label{sec:methodology}

To solve the given problem, we will use a similar approach adopted in the previous assignment.
The main differences are:

\begin{itemize}
    \item Adoption of the \textbf{Updated Lagrangian Formulation} instead of the Total Lagrangian Formulation.
    \item Different constitutive model (Truesdell objective rate).
    \item 2D problem instead of 1D.
\end{itemize}

As for the code implementation, a skeleton of the code may be summarized as follows:

\begin{enumerate}
    \item Definition of parameters for the problem (material properties, geometry of the plate, boundary conditions\dots).
    \item Mesh generator for the 2D plate (domain discretization, node numbering and connectivity dictionary).
    \item Explicit integration scheme with Updated Lagrangian Formulation.
    \item Plotting and post-processing of the results.
\end{enumerate}

\textbf{Note:} given the advice from the professor in the last lecture, for this assignment we have changed a bit the structure and the way of do things, so to rely less on the \texttt{MATLAB} built-in functions and more on the custom implementation of the code.
This will allow us to have a better understanding of the underlying mechanics of the problem and the numerical methods used to solve it.