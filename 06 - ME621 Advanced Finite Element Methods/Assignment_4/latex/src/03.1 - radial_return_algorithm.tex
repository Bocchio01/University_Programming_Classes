\subsection{Radial Return Algorithm}
\label{subsec:radial_return_algorithm}

The radial return algorithm is a method used to update the stress tensor in the plastic regime.

The algorithm is based on the assumption that the stress tensor is always on the yield surface, and the plastic strain is always normal to the yield surface.

The algorithm can be summarized in the following steps:

\begin{enumerate}
    \item Compute the trial stress tensor $\boldsymbol{\sigma}^{\text{trial}}$ using the constitutive model for the elastic regime.
    \item Compute the trial deviatoric stress tensor $\boldsymbol{s}^{\text{trial}}$.
    \item Compute the trial yield function $f^{\text{trial}}$.
    \item If $f^{\text{trial}} \leq 0$, the stress tensor is already on the yield surface, and the algorithm can proceed to the next step.
    \item If $f^{\text{trial}} > 0$, the stress tensor is outside the yield surface, and the algorithm needs to be corrected.
    \item Compute the plastic multiplier $\lambda$.
    \item Update the stress tensor using the plastic multiplier.
    \item Update the plastic strain tensor.
    \item Update the back stress tensor.
    \item Update the hardening variable.
    \item Update the stress tensor using the updated hardening variable.
\end{enumerate}