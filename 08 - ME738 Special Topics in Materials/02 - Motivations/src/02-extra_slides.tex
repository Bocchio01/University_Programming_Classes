\begin{frame}[standout]
    Extra slides
\end{frame}



\begin{frame}{Atomic clocks comparison}

    \begin{columns}[c, onlytextwidth]

        \begin{column}{0.25\textwidth}

            \begin{figure}[H]
                \centering
                \includegraphics[width=0.9\textwidth]{img/Microsemi-MHM-2020.png}
                \caption{Microsemi MHM 2020}
            \end{figure}

        \end{column}

        \begin{column}{0.25\textwidth}

            \begin{figure}[H]
                \centering
                \includegraphics[width=0.9\textwidth]{img/Microsemi-5071A.png}
                \caption{Microsemi 5071A}
            \end{figure}

        \end{column}

        \begin{column}{0.25\textwidth}

            \begin{figure}[H]
                \centering
                \includegraphics[width=0.9\textwidth]{img/Spectratime-iSpace-RAFS.png}
                \caption{Spectratime iSpace RAFS}
            \end{figure}

        \end{column}

        \begin{column}{0.25\textwidth}

            \begin{figure}[H]
                \centering
                \includegraphics[width=0.9\textwidth]{img/Microsemi-SA45s.png}
                \caption{Microsemi SA45s}
            \end{figure}

        \end{column}

    \end{columns}

    \begin{table}
        \centering

        \resizebox{\columnwidth}{!}{%
            \begin{tabular}{ll|lllllllllll}
                \hline
                \textbf{Vendor} & \textbf{Product} & \textbf{Type} & \textbf{ADEV} & \textbf{Aging} & \textbf{Tmin} & \textbf{Tmax} & \textbf{Tempco} & \textbf{Power} & \textbf{Weight} & \textbf{Size}           \\
                ~               & ~                & ~             & (1 s)         & (month)        & (°C)          & (°C)          & ~               & (W)            & (kg)            & (cm\textsuperscript{3}) \\
                \hline
                Microsemi       & MHM 2020         & Maser         & 8,00E-14      & 9,00E-15       & ~             & ~             & ~               & 75,00          & 246,000         & 374072                  \\
                Microsemi       & 5071A            & CBT           & 5,00E-12      & ~              & 0             & 55            & ~               & 50,00          & 30,000          & 29700                   \\
                Spectratime     & iSpace RAFS      & Space Rb      & 3,00E-12      & 8,30E-12       & -5            & 10            & ~               & 35,00          & 3,400           & 3224                    \\
                Microsemi       & SA45.s           & CSAC          & 3,00E-10      & 9,00E-10       & -10           & 70            & 1,00E-09        & 0,12           & 0,035           & 17                      \\
                \hline
            \end{tabular}
        }

        \caption{Key parameters}
    \end{table}

\end{frame}



\begin{frame}{GNSS (Trilateration)}

    \begin{columns}[c, onlytextwidth]

        \begin{column}{0.5\textwidth}

            \begin{figure}[H]
                \centering
                \includegraphics[width=0.9\textwidth]{img/GPS-Trilateration.png}
                \caption{GPS trilateration}
            \end{figure}

        \end{column}

        \begin{column}{0.5\textwidth}

            Basic math involved in trilateration:

            \begin{equation*}
                \Delta d_i = c \cdot \Delta t = c \cdot (t_{\text{sat}} - t_{\text{rcvr}})
            \end{equation*}

            Our system of equations has 4 unknowns:

            \begin{equation*}
                \underbrace{d_1, d_2, d_3, \Delta t}_\text{4 Unknowns requires 4 satellites}
            \end{equation*}

            \textbf{Final position strongly depends on $\Delta t$.}

        \end{column}

    \end{columns}

    \vspace{10pt}

    Because of better holdover capabilities, the use of a CSAC in the receiver allows to eliminate the need for the 4th satellite after the first clock calibration.

    A more accurate frequency of the receiver clock allows for a faster GNSS codes search and indeed a lower power consumption.

\end{frame}