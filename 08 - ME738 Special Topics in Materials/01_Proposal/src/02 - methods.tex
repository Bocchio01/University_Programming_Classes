\section{Methods}

For each of the clock sources mentioned in the introduction, the project will follow the same structure:

\begin{enumerate}
    \item Working principle: a detailed explanation of the working principle of the clock source, including the physical phenomena involved and the main equations that govern its operation.
    \item Simulations (when applicable): basic self-made simulations of the clock source, using \texttt{Matlab} or \texttt{COMSOL Multiphysics} (if available through a university license).
    \item State of the art: a review of the most recent advancements in the field (scientific papers driven section).
    \item Applications: a review of the most recent applications of the clock source, including the main challenges and the potential future developments (patents driven section).
\end{enumerate}

Given the time constraints of the course, the project will focus on the first two points for each clock source, with the third point being addressed only if time permits.

In general, given a time constraint of 4/5 weeks, the project will be divided as follows (tentative):

\begin{itemize}
    \item Week 1: Crystal oscillators
    \item Week 2: \acrshort{mems} resonators
    \item Week 2: \acrshort{mems} resonators
    \item Week 4: \acrshort{mems} Atomic Clocks (time permitting)
    \item Week 5: Final report writing
\end{itemize}
