\documentclass[titlesec]{araproposal}

\usepackage{float}
\usepackage{graphicx}
\usepackage[font={small,it}]{caption}

\hypersetup{
pdftitle={ME738 - Special Topics in Materials},
pdfsubject={Project proposal},
pdfauthor={Tommaso Bocchietti}
}

\makeglossaries
\newacronym{mems}{MEMS}{Micro Electromechanical Systems}
\newacronym{nems}{NEMS}{Nano Electromechanical Systems}
\newacronym{csac}{CSAC}{Chip-Scale Atomic Clock}

\title{Chip-Scale Atomic Clock Source Technology}

\piauthor{Tommaso Bocchietti, Department of Mechanical \& Mechatronics Engineering, University of Waterloo}

\contact{Tommaso Bocchietti, \email{tbocchie@uwaterloo.ca}}
\date{\today}

\newcommand{\semph}[1]{\textbf{\textit{#1}}}
\newcommand{\para}[1]{\textbf{#1.}\ }


\begin{document}

\graphicspath{{img/}}

\maketitle

\abstract{
  Given the increase amount of applications requiring precise timekeeping, the demand for high-performance and low-cost clock sources has grown significantly.
  The constraints over the size, weight, and power consumption of modern electronics have led to the development of \acrfull{mems} \& \acrfull{nems}.

  In this project, we aim to investigate a particular class of \acrshort{mems} composed by the \textit{\acrfull{csac}}.

  The focus will be mainly on the principles and the engineering of these devices, and their potential applications.
  By doing so, we will also underline the motivations behind the use of these devices, and how they differ from the traditional clock sources.
}

\keywords{\acrfull{csac}, Principles, Engineering, Applications, State-of-the-art}


\section{Introduction}
\label{sec:introduction}

Model rocketry is a popular hobby that involves the design, construction, and launch of small rockets powered by solid fuel engines.

Even though, at this scale, models are significantly simplified compared to full-scale rockets, they still can be used to study the principles of aerodynamics, propulsion, and flight dynamics.
One of the key parameters that affect the performance of a model rocket is the drag coefficient $C_d$.

The drag coefficient is a dimensionless quantity that characterizes the aerodynamic drag of an object moving through a fluid.
It is defined as the ratio of the drag force acting on the object to the dynamic pressure of the flow around the object.

\begin{equation}
    C_d = \frac{F_d}{\frac{1}{2} \rho v^2 A}
    \label{eq:drag_force}
\end{equation}

The drag coefficient of a model rocket depends mainly (but not only) on its shape, flow conditions, and properties of the fluid.
In general, it cannot be considered a constant value and may vary significantly during the flight of the rocket.

However, for a small model rocket with a 'classical design', it has been observed that the drag coefficient typically ranges between $0.7$ and $0.8$.
This value is based on empirical data and is used as a rule of thumb during the design and analysis phases.

In this project, we aim to analyze the drag coefficient of one of our model rockets using \texttt{Ansys Fluent}, a popular commercial software used for computational fluid dynamics (CFD) simulations.
Due to inexperience with the software and in general the world of simulations, we will have the necessity to make some strong assumptions and simplifications that might affect the accuracy of the results.

At the end of the project, we will try to validate the CFD simulation results against predictions made using another software for model rocket design and simulation, and against actual flight data collected during a previous launch.

The model rocket analyzed in this project is shown in Figure \ref{fig:model_rocket_img}.

\begin{figure}[H]
    \centering
    \includegraphics[width=.28\textwidth]{img/Rocket.jpg}
    \caption{Picture of the model rocket analyzed in this project. Photo of the author (July 21, 2023).}
    \label{fig:model_rocket_img}
\end{figure}
\section{Methods}
\label{sec:methods}

\paragraph{Research Methodology}
The research will be conducted through a literature review of scientific articles, conference papers and thesis if available.
The main sources of information will be the IEEE Xplore, ScienceDirect, and Google Scholar databases.
The search will be conducted using the following keywords: "Chip-Scale Atomic Clock", "Atomic Clock", "MEMS", "CSAC", "Vapor Cell", "Rubidium", "Cesium", "Microwave Cavity", "Laser Cooling", "Photon Detector", "Quartz Crystal Oscillator", "Electron Spin", "Electron Excitation", "Optical Lattice Clock", "Quantum Technologies".

During the research, we will try to annotate the most relevant papers and articles, that will be then used to write the final report.

\paragraph{Outline}
We leave here a general outline that will be used as a guide for the development of the project.

\begin{enumerate}
    \item Introduction: Discuss the need for precise timekeeping and the exigence of chip-scale atomic clocks.
    \item \textbf{Engineering of Chip-Scale Atomic Clocks}: Discuss the principles of operation. Note: It would be interesting to use simulation tools such as \texttt{COMSOL Multiphysics} to visualize the operating principles of these devices, but not knowing the software and its capabilities, I am not sure if it is applicable here.
          \begin{itemize}
              \item Vapour cell
              \item Magnetic selector (electron spin)
              \item Microwave cavity (electron excitation at hyperfine transition)
              \item Laser system (laser cooling and trapping of atoms)
              \item Photon detector
              \item Closed loop over quartz crystal oscillator
          \end{itemize}
          \begin{figure}[H]
              \centering
              \includegraphics[width=.6\textwidth]{img/atomic_clock_logic}
              \caption{Schematic of the simplified atomic energy level configuration \cite{KNAPPE2008571}}
          \end{figure}

    \item \textbf{Technology Comparison}: Introduce the different types of chip-scale atomic clocks. Compare and contrast the different chip-scale atomic clock technologies in terms of size, power consumption, accuracy, and suitability for various applications.
          \begin{itemize}
              \item Cesium based
              \item Rubidium based
          \end{itemize}
          % \item Fabrication and Manufacturing: Explore the components of chip-scale atomic clocks such as atomic vapor cells, laser systems, and control electronics. Understand the microfabrication techniques used in their manufacturing.
    \item Applications: Consider the diverse applications of chip-scale atomic clocks (aerospace and defense to telecommunications and scientific research)
    \item Challenges and Future Directions: Identify current challenges such as size reduction and power efficiency, and consider future trends like adoption of optical lattice clocks and integration with quantum technologies.
    \item Conclusion: Summarize the key findings of the research and discuss the implications for future advancements and applications of chip-scale atomic clocks.
\end{enumerate}

% In case of time availability, we will also cover the Fabrication and Manufacturing of these devices, understanding the microfabrication techniques used in their manufacturing.

\paragraph{Time schedule}
In general, given a time constraint of 4/5 weeks, the project will be divided as follows (tentative)

\begin{itemize}
    \item Week 1: Introduction + Engineering of Chip-Scale Atomic Clocks
    \item Week 2: Engineering of Chip-Scale Atomic Clocks (possible simulation and results analysis)
    \item Week 3: Technology Comparison + Applications
    \item Week 4: Challenges and Future Directions + Conclusion
    \item Week 5: Final report writing
\end{itemize}

\section{Expected Results}

Please include milestones with timeline estimates, such as for datasets, code
releases, technical reports, publications, applications, presentations, etc.


\appendix
\printglossary[type=\acronymtype]

\bibliographystyle{plainurl}
\bibliography{ref}

\end{document}
