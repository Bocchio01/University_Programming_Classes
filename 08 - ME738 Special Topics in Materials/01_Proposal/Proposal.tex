\documentclass[titlesec]{araproposal}

\usepackage{float}
\usepackage{graphicx}
\usepackage[font={small,it}]{caption}

\hypersetup{
pdftitle={ME738 - Special Topics in Materials},
pdfsubject={Project proposal},
pdfauthor={Tommaso Bocchietti}
}

\makeglossaries
\newacronym{mems}{MEMS}{Micro Electromechanical Systems}
\newacronym{nems}{NEMS}{Nano Electromechanical Systems}
\newacronym{csac}{CSAC}{Chip-Scale Atomic Clock}

\title{Chip-Scale Atomic Clock Source Technology}

\piauthor{Tommaso Bocchietti, Department of Mechanical \& Mechatronics Engineering, University of Waterloo}

\contact{Tommaso Bocchietti, \email{tbocchie@uwaterloo.ca}}
\date{\today}

\newcommand{\semph}[1]{\textbf{\textit{#1}}}
\newcommand{\para}[1]{\textbf{#1.}\ }


\begin{document}

\graphicspath{{img/}}

\maketitle

\abstract{
  Given the increase amount of applications requiring precise timekeeping, the demand for high-performance and low-cost clock sources has grown significantly.
  The constraints over the size, weight, and power consumption of modern electronics have led to the development of \acrfull{mems} \& \acrfull{nems}.

  In this project, we aim to investigate a particular class of \acrshort{mems} composed by the \textit{\acrfull{csac}}.

  The focus will be mainly on the principles and the engineering of these devices, and their potential applications.
  By doing so, we will also underline the motivations behind the use of these devices, and how they differ from the traditional clock sources.
}

\keywords{\acrfull{csac}, Principles, Engineering, Applications, State-of-the-art}


\section{Introduction}

Every electronic device requires the use of a timing system (usually referred as clock) that sets the reference for the operation of the device.
This timing system is implemented through the use of various electronic components such as clock source, phase comparator, frequency dividers, \dots.

The aim of this project is to investigate the state of the art of clock sources technologies, with a focus on the most recent advancements in the field and their potential applications.
In particular, the project will focus on the following topics:

\begin{itemize}
    \item \semph{Crystal oscillators}: electric oscillator type circuit that uses a piezoelectric resonator, a crystal, as its frequency-determining element (\href{https://en.wikipedia.org/wiki/Crystal_oscillator#Terminology}{Wikipedia definition}).
    \item \semph{\acrshort{mems} resonators}: small electromechanical structures that vibrate at high frequencies (\href{https://en.wikipedia.org/wiki/Microelectromechanical_system_oscillator#Resonators}{Wikipedia definition}).
    \item \semph{\acrshort{mems} Atomic Clocks} (time permitting): the combination of a \acrshort{mems} system fabrication with atomic clocks for small, cheap, low-power devices \cite{KNAPPE2008571}.
\end{itemize}



\section{Outline}
\label{sec:outline}

We leave here a general outline that will be used as a guide for the development of the project.

\begin{enumerate}
    \item Introduction: Discuss the need for precise timekeeping and the exigence of chip-scale atomic clocks.

    \item Engineering of Chip-Scale Atomic Clocks: Discuss the principles of operation and the advantages/disadvantages of each type. Possibility to use simulation tools such as \texttt{COMSOL Multiphysics} to visualize the principles of operation of these devices (based on the time available and the complexity of the simulation and the software).

    \item Technology Comparison: Introduce the different types of chip-scale atomic clocks, such as rubidium and cesium-based devices. Compare and contrast the different chip-scale atomic clock technologies in terms of size, power consumption, accuracy, and suitability for various applications.

          % \item Fabrication and Manufacturing: Explore the components of chip-scale atomic clocks such as atomic vapor cells, laser systems, and control electronics. Understand the microfabrication techniques used in their manufacturing.

    \item Applications: Investigate the diverse applications of chip-scale atomic clocks, ranging from aerospace and defense to telecommunications and scientific research.

    \item Challenges and Future Directions: Identify current challenges such as size reduction and power efficiency, and consider future trends like adoption of optical lattice clocks and integration with quantum technologies.

    \item Conclusion: Summarize the key findings of the research and discuss the implications for future advancements and applications of chip-scale atomic clocks.
\end{enumerate}

In case of time availability, we will also cover the Fabrication and Manufacturing of these devices, understanding the microfabrication techniques used in their manufacturing.

In general, given a time constraint of 4/5 weeks, the project will be divided as follows (tentative)

\begin{itemize}
    \item Week 1: Introduction + Engineering of Chip-Scale Atomic Clocks
    \item Week 2: Engineering of Chip-Scale Atomic Clocks (possible simulation and results analysis)
    \item Week 3: Technology Comparison + Applications
    \item Week 4: Challenges and Future Directions + Conclusion
    \item Week 5: Final report writing
\end{itemize}

\section{Expected Results}

The main expected result of the project is the implementation of the \acrshort{simple} algorithm.
We will report the mathematical derivation of the algorithm, the implementation details, and the results obtained for the Lid Driven Cavity problem in the same PDF report used also for the first assignment.

The second expected (and optional) result is the analysis of the velocity flow field around a 2D geometry.
We will report the results obtained considering different problem geometries and wind parameters (speed \& direction) in a separate PDF report.

Up to now, we don't have the competences to use a commercial \acrshort{cfd} software to compare our results with.
We would greatly appreciate it if the professor could provide us with some results obtained from such software for comparative analysis, if feasible.

In the end, we expect to have a better understanding of the \acrshort{cfd} solvers at least for 2D steady-state problems, and to be able to have a higher self-consciousness about their implementation and their limitations.

\appendix
\printglossary[type=\acronymtype]

% \bibliographystyle{plainurl}
% \bibliography{ref}

\end{document}
