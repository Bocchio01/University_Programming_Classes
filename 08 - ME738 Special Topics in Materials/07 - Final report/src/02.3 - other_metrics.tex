\subsection{Other metrics}
\label{subsec:other_metrics}

While stability is probably the most important parameter to define the performance of a clock, other metrics are also used to give a more complete overview of the device.
Some of the most important ones are:

\begin{itemize}
  \item Phase noise $\mathcal{L}(f)$: represents the noise power relative to the carrier contained in a 1 Hz bandwidth centered at a certain offset from the carrier. It's typically expressed in units of $dBc/Hz$.
  \item Temperature sensitivity $tempco$: measures how much the clock's performance varies with temperature changes. It's typically expressed in units of $ppb/^\circ C$.
  \item Operating temperature range: defines the minimum and maximum temperatures the clock can function within.
\end{itemize}

In the following sections, we will focus mainly on the stability, power consumption, and size of the \acrshort{csac}, given that the original purpose of this technology was to create a reliable, portable and low-power atomic clock.

