\subsection{Ocean Bottom Seismics (OBS)}
\label{subsec:OBS}

Another application that benefits from the \acrshort{csac} technology is the exploration of the Earth's crust and mantle done via Ocean Bottom Seismics (OBS) technique.
This technique consists of deploying a network of geophones on the ocean floor to measure the travel times of reflected or refracted seismic waves generated by earthquakes or artificially generated by air guns or explosives.
The data collected by the geophones are then used to create a 3D map of the Earth's crust and mantle.

The oil exploration industry is one of the main user of this technique, as it helps estimate the location and depth of various features of interest, such as oil reservoirs and gas deposits.


\paragraph{Role of CSAC}

In order to work properly, the timestamp of each geophones must be accurate and coordinated with the others in the networks.
However, the signal from the GNSS system is not be available on the ocean floor, and the traditional quartz oscillator might not be able to provide the required stability over time.

\acrshort{csac} instead, can be synchronized with the GNSS system before the deployment, and then work autonomously for weeks or even months on the ocean floor, providing a precise timestamp for the geophones.
This method allows for a more accurate seismic mapping than the traditional one, where geophones are placed some meters below the surface ocean allowing them to communicate with the GNSS system.
