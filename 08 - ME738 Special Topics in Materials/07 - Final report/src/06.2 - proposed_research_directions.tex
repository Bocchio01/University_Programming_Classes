\subsection{Proposed research directions}
\label{subsec:proposed_research_directions}

To overcome the current limitations and deviate from the relation between stability and SWaP explained in Section \ref{subsec:stability_vs_SWaP}, a different physics approach must be exploited.

After more than 15 years of development, three main research directions have been identified as the most promising for the development of the next generation of \acrshort{csacs}:

\begin{itemize}
    \item Microwave transitions in laser-cooled alkali metals.
    \item Microwave transitions in double-resonance trapped ions.
    \item Optical transitions in warm atomic/molecular vapors.
\end{itemize}

In the following subsections, we will analyze the current state of the art of these research directions and the issues and challenges that need to be addressed to make them suitable for commercial applications.


\subsubsection{Microwave transitions in laser-cooled alkali metals}
\label{subsubsec:laser_cooled_alkali}

The first research direction is based on the use of microwave transitions in laser-cooled alkali metals.

The use of cold atoms in the reference cell instead of warm atoms allows for a significant reduction in the Doppler broadening of the atomic transitions and the collisional shifts, which can be considered the main sources of frequency instability in the current \acrshort{csac} technology.

In the proposed architecture, both \acrshort{rb} and \acrshort{cs} atoms are laser-cooled and trapped in a 2D-MOT\footnote{2D Magneto-Optical Trap}, before being transferred to a dedicated interrogation cell.
Here two distinct CPT light fields are used at first to prepare the atoms in a coherent superposition of states, and then to probe the atomic transitions based on the local oscillator frequency.

From a stability point of view, the results look promising as the Doppler broadening is significantly reduced and the line quality factor is greatly improved.
The main bottlenecks however exist and are given by the technological and experimental limitations.
In particular, pressure drifts of the atom source, imperfections in the CPT optical implementation, and technical noise on the lasers are the main sources of instability, rather than fundamental physics.


\subsubsection{Microwave transitions in double-resonance trapped ions}
\label{subsubsec:double_resonance_ions}

The second research direction is based on the use of microwave transitions in double-resonance trapped ions.
Up to today, this seems to be the most promising path as it allows for a significant reduction in the Doppler broadening and the possibility of miniaturization.

The working principle is very similar to the one of the laser-cooled alkali metals, but with the use of ions instead of atoms and a different cooling mechanism.
In particular, both Ytterbium-171+ (Yb+) and Mercury-199+ (Hg+) based clocks have been developed and tested, showing excellent stability thanks to the Paul's trap mechanism.

A couple of fully functional prototypes have been developed, such as the Yb+ clock developed by Sandia National Laboratories and NASA's Jet Propulsion Laboratory (JPL) in 2015, and the Hg+ clock developed by JPL in 2019.
Both of them already met the stability requirements imposed by the DARPA programs, but still face some challenges related in particular to Stark effect.

The use of electric fields for ion manipulation can cause a shift in the ions' transition frequency.
To mitigate this, pulsed laser interrogation techniques are required, which indeed complicate the system design and the miniaturization of the components.

On the other hand, the simplicity of the trapping system and the high manipulability of the ions make this approach the most promising in the optics of the next generation of \acrshort{csacs}.


\subsubsection{Optical transitions in warm atomic/molecular vapors}
\label{subsubsec:optical_transitions}

The third research direction is based on the use of optical transitions in warm atoms environments.

The switching from a microwave range in the target transition frequency to a much higher energy (optical) range allow an increase in both the line quality factor and the signal-to-noise ratio, allowing for a significant improvement in the clock stability.

However, this approach is the most challenging from a technological point of view as it requires the development of a series of miniaturized components, such as Kerr-micro-resonator frequency combs, waveguides, and microfabricated photomultiplier tubes.
Those components are still in the experimental stage and require further development to be suitable for commercial applications.

At the moment, some earlier version prototypes have been developed, such as the NIST Chip-Scale Optical Atomic Clock in 2019 and the NIST-DRAPER Miniature Atomic Clock in 2020.
As in the case of the laser-cooled alkali metals or the double-resonance trapped ions, also these clock already met the stability requirements imposed by the DARPA programs, but still face some challenges related to the miniaturization of the components.


