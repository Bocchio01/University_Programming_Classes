\section{Introduction}
\label{sec:introduction}

\paragraph{The importance of precise timekeeping}

Accurate timekeeping has always been essential for human activities, from coordinating daily tasks to enabling global communication and navigation systems.
Over time, scientists have developed increasingly precise timekeeping devices, with the invention of atomic clocks marking a significant advancement in this field and setting new standards for accuracy and stability.

However, traditional atomic clocks were large, complex, and power-hungry devices, limiting their portability and widespread use.

\paragraph{The exigence of miniaturization}

In the early 2000s, the need for an ultra-miniaturized, low-power atomic time and frequency reference unit arose, driven by military applications from the Defense Advanced Research Projects Agency.
DARPA collaborated with the National Institute of Standards and Technology (NIST) to develop the first Chip Scale Atomic Clock (\acrshort{csac}) in 2005, which was a milestone achievement in atomic clock technology.

\paragraph{Current uses and future potential}

Leveraging the natural oscillations of atoms, \acrshort{csacs} are able to achieve frequency stability on the order of $10^{-11}$ over a day while consuming less than $100 mW$ of power and fitting into a volume of less than $1 cm^3$.

These remarkable performance metrics have enabled \acrshort{csacs} to find applications in a wide range of fields, including satellite navigation, telecommunications, and space exploration.
Moreover, the compact size and low power consumption of \acrshort{csacs} make them ideal for integration into portable devices, unmanned aerial vehicles (UAVs), and other applications where precise timekeeping is essential.

In this paper, we will try to understand the main concepts behind the operation of \acrshort{csacs}, what are the current limitations, and what we can expect from the future development of this technology.