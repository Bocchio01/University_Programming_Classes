\subsection{Long-term frequency stability}
\label{subsec:long_term_stability}

Long-term stability refers to the ability of maintaining frequency accuracy over extended periods (hours, days or even months).

Here, all the previously mentioned disturbances can have a significant impact.
The aging of the components is also relevant and to be considered.
Over longer periods of time, the pressure inside the reference cell might change, the quality of the excitation source might degrade, and so also the mechanical properties of the local oscillator.

Temperature exposition is also a key factor in the long-term stability of the clock.
If excessive temperature is applied for long periods of time, the gas pressure inside the reference cell will increase and also the vacuum integrity of the entire clock would be affected.
Moreover, the electronics circuits degrade faster at higher temperatures, reducing the overall lifetime of the clock.
