\section{Conclusion}
\label{sec:conclusion}

\acrfull{csacs} represent a significant advancement in the field of atomic clocks, offering a miniaturized and low-power alternative to traditional atomic clocks.
Their enhanced stability and accuracy thanks to the use of atomic transitions as a discipline reference for the local oscillator make them ideal for a wide range of applications, where precise timekeeping and size, weight, and power constraints are critical.

However, the physics enabling the operation of \acrshort{csacs} also poses significant challenges in a further step towards better performance and miniaturization.
Due to the intrinsic limitations, also the applications involving \acrshort{csacs} often relies of some external reference to correct the drift and the aging on the long term.

To overcome these limitations, the development of the next generation of \acrshort{csacs} is an active area of research.
After more than 15 years since the launch of the first program to develop the NG-CSACs, several technological solutions have been proposed, but a clear winner has yet to emerge.

The potential of NG-CSACs is vast, extending far beyond precise timekeeping.
In case of success, these devices have the potential to revolutionize fields like microfabrication, quantum computing, and even our understanding of fundamental physics phenomena.
We believe that pursuing the development goals set for NG-CSACs represents an investment in the future, paving the way for groundbreaking advancements across various scientific and technological disciplines.