\section{Conclusions}
\label{sec:conclusions}

\acrfull{csacs} represent a significant advancement in the field of atomic clocks, offering a miniaturized and low-power alternative to traditional table-size atomic clocks.
Their enhanced stability and accuracy, thanks to the use of atomic transitions as a discipline reference for the local oscillator, make them ideal for a wide range of applications where precise timekeeping and size, weight, and power constraints are critical.

However, the physics enabling the operation of \acrshort{csacs} also poses significant challenges in a further step towards better performance and miniaturization.
Due to their well known limitations, the applications involving \acrshort{csacs} often relies on some external reference to correct the drift and the aging over long-term period.

To overcome these limitations, the development of the \acrfull{ngcsacs} has been launched and is currently an active area of work.
After more than 15 years since the activation of the first program to develop \acrshort{ngcsacs}, several technological solutions have been proposed, but a clear winner hasn't emerged yet.

The potential of \acrshort{ngcsacs} is vast, extending far beyond precise timekeeping.
In case of success, these devices have the potential to revolutionize fields like microfabrication, quantum computing, and even our understanding of fundamental physics phenomena.
We believe that pursuing the development goals set for \acrshort{ngcsacs} represents an investment in the future, paving the way for groundbreaking advancements across many scientific and technological disciplines.